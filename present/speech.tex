\documentclass{ctexrep}


% Title Page
\title{Security Matching System Speech}
\author{Shen Guanzhi}


\begin{document}
\maketitle

\chapter{问题分析}
\section{市场周期}
证券交易所的一个交易日内会呈现$4$种不同的状态,这里简单地将其称作睡眠状态、准备状态、集合竞价状态和连续竞价状态。其中,从集合竞价状态切换到睡眠状态时需要计算出开盘或收盘交易。图中做了一定的简化,忽略了连续竞价中场休息等状态切换过程。

\section{单次撮合过程}
集合竞价和连续竞价的基础都是单次撮合算法——即比较并修改最优买单和最优卖单。

\section{集合竞价与连续竞价对比}
尽管都是对于单次撮合算法的循环,集合竞价算法与连续竞价算法却需要完全不同的实现。

计算开盘价与收盘价的任务是“不可抢占的”。这里的“不可抢占”不是说它们对于操作系统来说是原子操作,而是强调它们内部独占所需的数据资源,不必考虑任何任务间数据竞争与同步的问题。由于参与集合竞价的订单可能很多,计算结盘价的任务应该视为CPU密集型的任务。为了公平性,不同证券的结盘价计算任务需要尽量并发进行。本系统中直接采用了第三方库来进行此调度工作。

连续竞价任务则相反。过程中用户随时可以有查询行情或撤销订单的操作。如果竞价任务长时间占据数据资源,会使得查询或撤单任务陷入饥饿,违反了公平性原则。

\chapter{程序设计}
\section{买卖盘}
每支证券对应一个买卖盘。一个买卖盘包含一个买单队列和一个卖单队列。

系统的核心任务是“撮合”,因此订单队列需要是优先队列。优先队列的实现方式可以是二叉堆——就是用完全二叉树实现的那种数据结构;也可以是平衡搜索树——包括AVL树、红黑树这些。

比较可知,二叉堆的优势在于查询但不删除最优元素的效率更高。但是,由于买单和卖单在统计学上总体是平衡的,这种“只查询但不删除”的操作并没有那么频繁。

平衡树还有其他的优势——比如能够直接获取排好序的即时行情。

尽管最终版本没能实现,本系统设计之初曾试图考虑过“订单时刻完全相同”的问题。在这种情况下,平衡树将是唯一正确的选择。因此,由于更加优秀的综合性能,平衡树是比二叉堆更为合适的订单队列数据结构。

\section{撮合器}
\subsection{组成}
单独一个买卖盘是无法撮合订单的,需要一些其他数据结构辅助。等待队列用于缓存那些已生成订单号但尚未进入买卖盘的订单——接收订单并生成订单号是$O(1)$的,但插入买卖盘并竞价则可能遇到先前订单占用的问题,需要等待更多时间。“竞价”任务是一个单独的协程,用于完成等待中订单的入盘。Watcher之后再介绍。

\subsection{难点}
$1$个撮合器会受到$5$种不同协程的读写,各自之间有着较为复杂的竞争与同步关系。这些关系是整个撮合算法的核心与难点。“接收”任务需要唤醒“竞价”任务并挂起“计算结盘价”任务;“竞价”任务需要在清空等待队列后唤醒“计算”任务并挂起自身,“撤销”任务与“获取行情”任务都需要与其他任务竞争买卖盘的读写权。

\subsection{同步方案}
整个撮合算法力求兼顾正确性与公平性。买卖盘和等待队列使用读写锁保护。Watcher是一个特殊管道,其中有且只有一个布尔值。

下面这个图是为了示意撮合过程中状态变化的复杂性的,不需要细看。编程实现其实并不复杂——“竞价”任务等待Watcher值为true时试图清空等待队列,队列空后将Watcher置false;“计算”任务等待Watcher值为false时试图锁定买卖盘并结算。

\chapter{数据库设计}
A/C库存储账户数据。Req(uest)库记录买单、卖单、撤单请求,其主键兼作订单号,随时间递增。Order库用于备份买卖盘,以防宕机。Rec(ord)库记录成交记录。Msg库用于备份未发出的用户通知。

数据库选用了 PostgreSQL ,程序中利用了它的事务机制以保证了交易的一致性。

\end{document}          
